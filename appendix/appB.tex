% -*-coding: utf-8 -*-

\BiAppChapter{算法}{Algorithms and Codes}

%=========================================================================================
% \BiSection{算法}{Algorithms}

% 算法描述使用 algorithm2e 宏包,效果如算法~\ref{alg_appB_lms}~所示。

% \begin{algorithm}[ht]
% 	\caption{LMS~算法详细描述 \label{alg_appB_lms}}
% 	\IncMargin{2em}
% 	\DontPrintSemicolon
% 	\KwIn{$\mbf{x}(k), \quad \mu, \quad \mbf{w}(0)$}
% 	\KwOut{$y(k), \quad \varepsilon(k)$}
% 	%
% 	\For{$k=0,1,\cdots$}{
% 		$y(k) = \mbf{w}^H(k)\mbf{x}(k)$ \tcp*{output signal}
% 		$\varepsilon(k) = d(k)-y(k)$ \tcp*{error signal}
% 		$\mbf{w}(k+1) = \mbf{w}(k)+\mu\varepsilon^{\ast}(k)\mbf{x}(k)$ \tcp*{weight vector update}
% 	}
% \end{algorithm}


%=========================================================================================
\BiSection{代码}{Codes} 

源代码使用 listings 宏包,LMS~算法的~Verilog~模块端口声明如代码~\ref{code_appB_lms}~所示。

{\fontsize{10pt}{0.5\baselineskip}\selectfont
	\begin{lstlisting}[caption={空时~LMS~算法~Verilog~模块端口声明},label={code_appB_lms}]
int main()
{

	std::string root_path = "E:\\Code\\heils_mobileface\\heils_mobileface\\heils_mobileface\\images\\lack";
	std::string suffix = ".jpg";
	
	std::vector<std::string> fns;
	glob(fns, root_path, suffix);
	//cout << pattern << endl;
	std::vector<cv::Mat> images;

	size_t face_num = 0;
	MtcnnDface facedetector = MtcnnDface();

	facedetector.initModel();
	for (int i = 0; i < fns.size(); i++) {
		cv::Mat cv_mat = cv::imread(fns[i]);
		ncnn::Mat ncnn_mat = ncnn::Mat::from_pixels(cv_mat.data, ncnn::Mat::PIXEL_BGR2RGB, cv_mat.cols, cv_mat.rows);

		std::vector<Bbox> finboxes;
		ptimer pt;
		//double start = static_cast<double>(cv::getTickCount());
		facedetector.detect(ncnn_mat, finboxes);
		//Sleep(1000);
		double inference = cv::getTickCount();
		std::cout << "boost timer:" << pt.elapsed() << "秒" << std::endl;
		//std::cout << "opencv timer:" << (inference - start) / cv::getTickFrequency() << "秒" << std::endl;
		

		if (finboxes.size() != 0)
			face_num++;
		for (auto &box : finboxes)
		{
			cv::rectangle(cv_mat, cv::Rect(box.x1, box.y1, box.x2 - box.x1 + 1, box.y2 - box.y1 + 1), cv::Scalar(0, 255, 255), 2);
			for (int j = 0; j < 5; ++j)
			{
				cv::circle(cv_mat, cvPoint(box.ppoint[j], box.ppoint[j + 5]), 2, CV_RGB(0, 255, 0), CV_FILLED);
			}
		}
		cv::imshow("show", cv_mat);
		cv::waitKey(0);
	}
	std::cout<<"image count" << fns.size() <<"face num:"<< face_num << std::endl;
	
  	facedetector.releaseModel();
	
	return 0;
}

	\end{lstlisting}
}

% 在最后一个附录 tex 文件末尾添加以下命令,使研究成果从奇数页开始
\clearpage{\pagestyle{empty}\cleardoublepage}
